\documentclass[a4paper, 11pt]{exam}
\usepackage{titling}
\usepackage{float}
\newcommand{\subtitle}[1]{%
  \posttitle{%
    \par\end{center}
    \begin{center}\large#1\end{center}
    }%
}

\usepackage{url}
\usepackage{amsmath,amsthm,enumitem,amssymb}
\usepackage{graphicx}
\usepackage{hyperref}
\renewcommand{\labelenumii}{\roman{enumii}}
\usepackage{courier}


\title{Homework Assignment 6}
\subtitle{CS/ECE 6810: Computer Architecture \\
Marek Baranowski\\
April 11, 2018}

\author{ \\
\textbf{Main Memory}}
\date{Due Date: April 18, 2018.\\
(90 points)}

\begin{document}
\maketitle
\begin{center}
\fbox{\fbox{\parbox{5.9in}{\centering
Important Notes:
\begin{itemize}
\item Solutions turned in must be your own. Please, mention references (if any) at the end of each question. Please refrain from cheating.
\item All solutions must be accompanied by the equations used/logic/intermediate steps and assumptions. Writing only the final answer will receive \textbf{zero} credit.
\item All units must be mentioned wherever required.
\item Late submissions \textbf{(after 11:59 pm on 04/18/2018)} will not be accepted
\item All submissions must be in PDF only. Scanned copies of handwritten solutions will not be accepted as a valid submission.
\end{itemize}
}}}
\end{center}
\vspace{0.2in}


\begin{enumerate}
 \item \textbf{Memory Scheduling Mechanisms (20 points): } For the following memory access pattern, estimate when each memory access completes for two different scheduling mechanisms: \textbf{open-page policy} and \textbf{close-page policy}. You are allowed to reorder requests already waiting in the memory controller. The access pattern only specifies the row being touched. All accesses are to the \textbf{same} bank. Assume that bus latencies are zero. Assume that the bank is already pre-charged at time 0. Assume that pre-charge takes 20 ns, loading a row buffer takes 20 ns, and cache line transfer to output pins also takes 20 ns.\newline


\begin{center}
\footnotesize
\begin{tabular}{|c|c|c|c|}
 \hline
 \textbf{Row being} & \textbf{Arrival time at} & \textbf{Open-page} & \textbf{Close-page} \\
  \textbf{accessed} & \textbf{memory controller} & & \\
 \hline
 X  & 10 ns & 50ns (RAS + CAS) & 50ns for data \\
    &       &                  & Next command at 70ns\\
 \hline
 X  & 70 ns & 90ns (CAS)  & 110ns for data\\
    &       &             & Next command at 130ns\\
 \hline
 Y  & 90 ns & 150ns (Pre + RAS + CAS) & 170ns for data\\
    &       &                  & Next command at 190ns\\
 \hline
 X  & 100 ns & 210ns (Pre + RAS + CAS) & 230ns for data\\
    &       &                  & Next command at 250ns\\
 \hline
 Y  & 180 ns & 270ns (Pre + RAS + CAS)  & 290ns for data\\
    &       &                  & Next command at 310ns\\
 \hline
 \end{tabular}
\end{center}
The scheduling in the open page policy doesn't quite line up to allow us
to reorder loads, hence we have to do row cycling on the last 3 accesses.

The closed page policy always does a RAS and CAS for the load, then a precharge,
effectively the next command cannot start until 70ns after the first one was
issued. The closed page policy could be reordered, e.g., the last X 
could come before the first Y, but it wouldn't really affect the final total
time for all memory accesses to complete.
 
\item \textbf{Memory System Design (20 points): } Modern systems have processors with four memory channels. Assume that a memory channel has 64 bits for data and 24 bits for address and command.
Assume that a memory channel is connected to eight x8 memory chips. Consider a new memory system that shrinks the data bus of the memory channel to 32 bits. Assume that such a channel would be connected to four x8 memory chips. Mention \textbf{two pros} and \textbf{two cons} of the new memory system (Be very concise).\newline

Pros:
\begin{enumerate}
\item There are fewer wires connecting the DRAM to the CPU.
\item A single memory access has lower latency.
\end{enumerate}

Cons:
\begin{enumerate}
\item Lower bandwidth on the bus.
\item The number of DRAM commands will be doubled.
\end{enumerate}
 
\item \textbf {Virtually Indexed Physically Tagged Cache (20 points): }  Assume that the OS uses a minimum page size of 4 KB. Assume that your L1 cache must be 8-way set-associative. If you're trying to correctly implement a virtually indexed physically tagged cache (with no additional support from the OS), what is the largest L1 cache that you can design? \newline

  We have 4KB = $2^{12}$B pages, so our cache index and offset are 12
  bits from the VA. We have 8 entries per set giving the formula
  $cache_{size}= 8\cdot2^{index}\cdot2^{offset}$ since
  $index + offset = 12$, we have a cache of size
  $2^{15}$B=32KB, regardless of the number of sets or cache line width.

\item \textbf {DRAM Control (10 points): } Express the latency of a memory load as a function of tRAS, tRP, tRCD, tCL in the following situations:

 \begin{enumerate}
 \item  The correct row is already placed in the row buffer

   If there is a row hit, we need only do a CAS. Here we have {\it
     t}CL which is the time for the row buffer to be readable
   (volatages are stable enough to know the content) after a CAS. The
   time in this situation is {\it t}CL. (I referenced 
   \begin{verbatim} https://en.wikipedia.org/wiki/Dynamic_random-access_memory\end{verbatim}
   for this).

 \item  Another row is already placed in the row buffer.

In the case of a full row and a miss, we need to do a precharge, activate and
CAS. The commands we issue are Row Precharge (RP), RAS and CAS, this takes time
{\it t}RP + {\it t}RCD + {\it t}CL.
 \end{enumerate}

\item \textbf {DRAM Control Tasks – Request Scheduling (20 points): }  Find the total number of commands sent by an in-order command scheduler for the given sequence of memory addresses for reads, using the following address mapping scheme:

\hspace{20pt} Address = row(12):bank(3):rank(1):channel(0):column(16)\newline
\hspace{20pt} Assume that all banks are initially precharged.

\center Addresses: \newline
00040108\newline
01040101\newline
00161804\newline
01040104\newline
\begin{table}[H]
\small
\begin{center}
\begin{tabular}{|c|c|c|c|c|c|}
\hline
Address & row(13) & bank(3) & column(16) & Commands & Reason\\
\hline
0x00040108 & 0000000000000 & 100 & 0000000100001000 & Activate, Read & Row buffer empty\\
\hline
0x01040101 & 0000000100000 & 100 & 0000000100000001 & Precharge, Activate, Read & Row miss\\
\hline
0x00161804 & 0000000000010 & 110 & 0001100000000100 & Activate, Read & Row buffer empty\\
\hline
0x01040104 & 0000000100000 & 100 & 0000000100000100 & Read & Row hit\\
\hline
\end{tabular}
\end{center}
\end{table}%
Eight commands are needed for this sequence of memory accesses.
\end{enumerate}
\end{document}
