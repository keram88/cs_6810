%%% Template originaly created by Karol Kozioł (mail@karol-koziol.net) and modified for ShareLaTeX use

\documentclass[a4paper,11pt]{article}

\usepackage[T1]{fontenc}
\usepackage[utf8]{inputenc}
\usepackage{graphicx}
\usepackage{xcolor}

\usepackage[tx]{sfmath}
\renewcommand\familydefault{\sfdefault}
\usepackage{tgheros}

\usepackage{amsmath,amssymb,amsthm,textcomp}
\usepackage{enumerate}
\usepackage{multicol}
\usepackage{tikz}

\usepackage{geometry}
\usepackage{float}
\geometry{total={210mm,297mm},
left=25mm,right=25mm,%
bindingoffset=0mm, top=20mm,bottom=20mm}


\linespread{1.3}

\newcommand{\linia}{\rule{\linewidth}{0.5pt}}

% custom theorems if needed
\newtheoremstyle{mytheor}
    {1ex}{1ex}{\normalfont}{0pt}{\scshape}{.}{1ex}
    {{\thmname{#1 }}{\thmnumber{#2}}{\thmnote{ (#3)}}}

\theoremstyle{mytheor}
\newtheorem{defi}{Definition}

% my own titles
\makeatletter
\renewcommand{\maketitle}{
\begin{center}
\vspace{2ex}
{\huge \textsc{\@title}}
\vspace{1ex}
\\
\linia\\
\@author \hfill \@date
\vspace{4ex}
\end{center}
}
\makeatother
%%%

% custom footers and headers
\usepackage{fancyhdr}
\pagestyle{fancy}
\lhead{}
\chead{}
\rhead{}
\cfoot{}
\rfoot{Page \thepage}
\renewcommand{\headrulewidth}{0pt}
\renewcommand{\footrulewidth}{0pt}
%

% all section titles centered and bolded
\usepackage{sectsty}
\allsectionsfont{\centering\bfseries\large}
%
% add section label
\renewcommand\thesection{\arabic{section}.}
%

%%%----------%%%----------%%%----------%%%----------%%%

\begin{document}

\title{CS 6810 - Performance Metrics}

\author{Marek Baranowski}

\date{01/30/2018}

\maketitle
\section{Summarizing performance numbers}
\begin{table}[H]
\caption{Expected Execution Time}
\begin{center}
\begin{tabular}{|c|c|c|c|c|c|c|}
\hline
& A & B & C & D & Sum & expected\\
\hline
BASE & 3s & 2.5s & 1s & 12s & 18.5s & ${18.5}\over{4}$s\\
NEW1 & 7s & 3s & 5s & 1s & 16s & $16\over4$s\\
NEW2 & 2s & 1s & 3s & 8s & 14
s & $14\over4$s\\
NEW3 & 1s & 3s & 2s & 13s & 19s & $19\over4$s\\
\hline
\end{tabular}
\end{center}
\label{default}
\end{table}%

\begin{table}[H]
\caption{Expected Energy}
\begin{center}
\begin{tabular}{|c|c|c|c|c|c|c|c|}
\hline
& A & B & C & D & Sum & expected energy & expected power\\
\hline
BASE & 20j & 40j & 50j & 15j & 125j & ${125}\over{4}$j&${125j\over18.5s}\approx6.8$W\\
NEW1 & 10j & 30j & 15j & 30j & 75j & ${75}\over{4}$j&${75j\over16s}\approx4.7$W\\
NEW2 & 30j & 60j & 20j & 20j & 130j & ${130}\over{4}$j&${130j\over14s}\approx9.3$W\\
NEW3 & 70j & 35j & 30j & 10j & 145j & ${145}\over{4}$j&${145j\over19s}\approx7.6$W\\
\hline
\end{tabular}
\end{center}
\label{default}
\end{table}%

\begin{enumerate}[i.]
\item We can see from Table 1 that system NEW2 gives the least execution time in expectation, since each program is equally likely to run.
\item From Table 2 we can see that NEW1 uses the least amount of expected energy.
\item In Table 2 we computed expected power by dividing total energy by total time for each system. From the table we can see NEW1 uses the least power.
\end{enumerate}

\section{Optimizing CPU time}
%%%%%%%%%%%%%%%%%%%%%%%%%%%%%%%%%%%%%%%%%%%%%%
\begin{table}[H]
\caption{New Instruction Frequencies (in percent)}
\begin{center}
\begin{tabular}{|c|c|c|c|}
\hline
&Original & Fused & Fused Normalized\\
\hline
Load & 10 & 10 & 10/70*100\\
Store & 5 & 5 & 5/70*100\\
Branch & 5 & 5 & 5/70*100\\
Add & 30 & 0 & 0\\
Mult & 50 & 20 & 20/70*100\\
FMA & 0 & 30 & 30/70*100\\
\hline
\end{tabular}
\end{center}
\label{default}
\end{table}%
Since 60\% of Mult instructions are able to be fused with Add, then 30\% of the total instructions are Mult instructions which can be merged with an Add instruction. Table 3 shows the original counts, the new raw counts and the normalized counts.
\begin{enumerate}[i.]
\item Original IPC is $\frac{1}{(.1*2+.05*1+.05*2+.3*1+.5*4)C} \approx .378$ instructions per cycle.\\ The new IPC is $\frac{1}{(10/70*2+5/70*1+5/70*2+0*1+20/70*4+30/70*4)C} \approx .298$ instructions per cycle, after accounting for the new machine using 70\% of the original instructions.
\item The speed up is $ \frac{OldTime}{NewTime}=\frac{OldCPI*CT}{NewCPI*.7*CT}=\frac{2.65 CT}{3.36*.7 CT}=1.13$ after accounting for the shorter program length. (CPI is simply 1/IPC.)
\end{enumerate}

\section{Amdahl's law}
%%%%%%%%%%%%%%%%%%%%%%%%%%%%%%%%%%%%%%%%%%%%%%
To use Amdahl's law here, we replace the $Speedup_{enhanced}$ portion with our enhanced improvement. The formula is
\begin{align*}
Improved_{overall} = \frac{1}{(1-Fraction_{enhanced}) + \frac{Fraction_{enhanced}}{Improved_{enhanced}}}.\\
\end{align*}
\begin{enumerate}[i.]
\item For reducing wireless we get $Improved_{enhanced} = \frac{1}{.9}$ since we're reducing by 10\%. The Fraction enhanced is 50\%. Plugging this in we get\\
$Improved_{Wireless} = \frac{1}{(1-.5) + \frac{.5}{1/.9}}\approx1.05$
\item For reducing CPU energy, $Improved_{enhanced}=\frac{1}{.4}$ by reduction by .6. The fraction enhanced is $.1$ giving:\\
$Improved_{CPU} = \frac{1}{(1-.1) + \frac{.1}{1/.4}}\approx1.06$
\item And reducing display energy by .5 gives $Improved_{enhanced}=\frac{1}{.5}$ and the fraction enhanced is $.2$, thus:\\
$Improved_{Graphics} = \frac{1}{(1-.2) + \frac{.2}{1/.5}}\approx1.11$
\end{enumerate}
Clearly we should enhance the Graphics system since it gives the largest savings.

\section{Power and energy}
%%%%%%%%%%%%%%%%%%%%%%%%%%%%%%%%%%%%%%%%%%%%%%
\begin{enumerate}[i.]
\item By default the machine uses (70+30)W=100W during execution, and the program takes 15s. Using this, we get 100W*15s=1500J of energy used.
\item Since we are reducing only the frequency by 30\% we will use 70*.7=49W of dynamic power. However, the program will take 15/.7=21.4s to run. Using the energy equation we get total energy is (49+30)W*21.4s=1691J.
\item If we reduce frequency and voltage by 30\% then static power becomes 30*.7W=21W and dynamic power is now $70*.7^2*.7W=24W$. The runtime is the same as the previous problem. Together this gives a total energy consumption of (21+24)W*21.4s=963J, which is a reduction from the first problem.
\end{enumerate}



\section{ISA}
%%%%%%%%%%%%%%%%%%%%%%%%%%%%%%%%%%%%%%%%%%%%%%
The following table shows what each instruction does by expanding out what the instruction does with memory and addressing.
\begin{table}[H]
\begin{center}
\begin{tabular}{|c|c|}
\hline
Instruction & effect\\
\hline
LOAD R5, 6000(R0) & R5 = 1 = mem[7000] = mem[6000+1000] = mem[6000+R0]\\
ADD R4, (R4) & R4 = 6012 = 6000 + 12 = 6000 + mem[6000] = R4 + mem[R4]\\
SUB R2, R1 & R2 = 74 = 99-25 = R2 - R1\\
LOAD R6, @(R0) & R6 = 33 = mem[3000] = mem[mem[R0]]\\
ADD R6, R4 & R6 = 6045 = 33 + 6012 = R6 + R4\\
SUB R5, R6 & R5 = -6044 = 1 - 6045 = R5 - R6\\
ADD R2, R5 & R2 = -5970 = 74 + (-6044) = R2 + R5\\
ADD R2, (R3 + R0) & R2 = -5899 = -5970 + 71 = R2 +mem[5000] = R2 + mem[R3+R0]\\
\hline
\end{tabular}
\end{center}
\label{default}
\end{table}%



\end{document}
