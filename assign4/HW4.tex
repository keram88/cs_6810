\documentclass[a4paper, 11pt]{exam}
\usepackage{titling}
\newcommand{\subtitle}[1]{%
  \posttitle{%
    \par\end{center}
    \begin{center}\large#1\end{center}
    }%
}

\usepackage{url}
\usepackage{amsmath,amsthm,enumitem,amssymb}
\usepackage{graphicx}
\usepackage{hyperref}
\renewcommand{\labelenumii}{\roman{enumii}}
\usepackage{courier}
\usepackage{float}


\title{Homework Assignment 4}
\subtitle{CS/ECE 6810: Computer Architecture \\
Marek Baranowski\\
March 14, 2018}

\author{ \\
\textbf{Memory Hierarchy: Cache}}
\date{Due Date: March 27, 2018.\\
(100 points)}

\begin{document}
\maketitle
\begin{center}
\fbox{\fbox{\parbox{5.9in}{\centering
Important Notes:
\begin{itemize}
\item Solutions turned in must be your own. Please, mention references (if any) at the end of each question. Please refrain from cheating.
\item All solutions must be accompanied by the equations used/logic/intermediate steps and assumptions. Writing only the final answer will receive \textbf{zero} credit.
\item All units must be mentioned wherever required.
\item Late submissions \textbf{(after 11:59 pm on 03/27/2018)} will not be accepted
\item All submissions must be in PDF only. Scanned copies of handwritten solutions will not be accepted as a valid submission.
\end{itemize}
}}}
\end{center}
\vspace{0.2in}


\begin{enumerate}
 \item \textbf{Cache Hierarchy (20 points): } Consider a processor with the following memory
organization: L1 Cache, L2 Cache, L3 Cache and main memory. Each cache stores both tags
and data. The data and tag arrays are going to be accessed with the same index value. Assume
that the processor does \textbf{serial} tag/data look-up (first tag lookup and then data access) for L2
and L3 caches and \textbf{parallel} tag/data look-up for L1 cache. The table given below provides the
time take to access the tag and data arrays in cycles
\newline


\begin{center}
\begin{tabular}{|c|c|c|c|}
 \hline
 \textbf{Cache level} & \textbf{Tag Access} & \textbf{Data Access} & \textbf{Hit Rate} \\
 L1 (32 KB)  & 1 & - & $50\%$ \\
 \hline
 L2 (1 MB) & 3 & 18 & $55\%$ \\
 \hline
 L3 (8 MB) & 25 & 85 & $75\%$ \\
 \hline
 Main Memory (8 GB) & - & 440 & - \\
 \hline
\end{tabular}
\end{center}

Find the number of cycles required to complete 2000 load instructions accessing this
hierarchy. \newline

First we find AMAT, which will allow us to determine the expected number of cycles for the 2,000 loads to complete.
Since this doesn't exactly follow the simplified formula given in the slides, we need to find the probability of getting data at each level, along with
the time to get the data to the CPU; since retrieving data from each level is mutually exclusive from getting it from a previous level, we can simply sum
the expected cycles for each level.
\begin{itemize}
\item L1: We have an expected time of $0.5\cdot1$ cycles for data to be available on a hit.
\item L2: On an L1 miss, 1 cycle has passed, so on an L2 hit, we expect $(1-0.5)\cdot0.55\cdot(1+3+18) = 0.5\cdot0.55\cdot22$ cycles for data to be available.
\item L3: On an L2 miss, 4 cycles have passed, so on an L3 hit, we expect $(1-0.5)\cdot(1-0.55)\cdot0.75\cdot(4+25+85) = 0.5\cdot0.45\cdot0.75\cdot114$ cycles for data to be available.
\item Mem: On an L3 miss, 29 cycles have passed, so when we reach this level, we expect $(1-0.5)\cdot(1-0.55)\cdot(1-0.75)\cdot(29+440) = 0.5\cdot0.45\cdot0.25\cdot469$ cycles for data to be available.
\end{itemize}

Adding these (and reducing) we get AMAT=$0.5\cdot 1 + 0.5\cdot(1+0.55\cdot(3+18)+0.45\cdot(3+0.75\cdot(25+85)+0.25\cdot(25+440)))=52.16875$ cycles for one load, so we expect 2000 load instructions to take $2000\cdot52.16875=104,337.5$ cycles.

 
\item \textbf{Performance Metric (20 points): } Consider a processor with a cache that has a hit time
of 5 cycles. If a miss occurs, then the penalty would be 150 cycles. Calculate AMAT if the
miss rate of this cache is $20\%$.\newline

AMAT = $5(t_h) + 0.2(r_m)\cdot150cycles(t_p) = 35cycles$

\item \textbf {Cache Addressing (30 points): }  Consider a processor using 2 cache levels. Level-1 cache is a
32KB direct mapped cache with 16B blocks used for both instructions and data. Level-2 is a
1MB, 4-way set associative cache with 64B cache lines. Assume that the processor can
address up to 8GB of main memory. 


 \begin{enumerate} [label=(\alph*)]
 \item  Find the number of bits for tag, index, and offset for level-1 and level-2 caches

 \begin{itemize}
 	\item First the processor can address $8\cdot2^{30}$ bytes of memory, so the address is 33 bits.
	\item For L1 we find:
	\begin{itemize}
	\item The blocks are 16=$2^4$ bytes, so the offset is 4 bits.
	\item The cache has $\frac{2^5\cdot2^{10} bytes}{2^4 bytes} = 2^{11}$ entries, so the index is 11 bits.
	\item Finally, we have $33-4-11=19$ bits for the tag.
	\end{itemize}
	\item For L2 we find:
	\begin{itemize}
	\item Each cache line is $64=2^6$ bytes, so the offset is 6 bits.
	\item From the 4 sets, and the 64B cache line we find $\frac{2^{20}bytes}{4\cdot64bytes}=\frac{2^{20}}{2^8}=2^{12}$ index locations, so the index is 12 bits.
	\item Finally the tag is $33-6-12=15$ bits.
	\end{itemize}
 \end{itemize}
 \item  Compute the size of the data and tag arrays in KB for each level. 
 \end{enumerate}

\item \textbf {Cache Replacement Policies (30 points): }  Consider a 2-way set associative cache with two
sets. When running two different programs on this machine, the following access patterns are
observed. 

\begin{itemize}
\item Program 1: C,A,B,D,B,F,C,E,A,D,B,F,A,B,C,E,B,A,F,D
\item Program 2: D,F,C,B,A,A,F,C,D,D,A,B,A,B,C,E,B,A,B,D
\end{itemize}
Assuming that blocks A, B, and C are mapped to set 1 and blocks D, E, and F are mapped to
set 2, compute the miss rates for each program (running solely) when Ideal, LRU, and MRU
replacement policies are applied.

In the following tables an `x' indicates a miss, and a `v' indicates a hit.

\begin{table}[H]
\caption{Program 1}
\begin{center}
\tiny
\begin{tabular}{|c|c|c|c|c|c|c|c|c|c|c|c|c|c|c|c|c|c|c|c|c|}
\hline
Ideal & C & A & B & D & B & F & C & E & A & D & B & F & A & B & C & E & B & A & F & D \\
\hline
Set 1, Way 1 & Cx &  &  &  &  &  & Cv &  & Ax &  &  &  & Av &  & Cx &  &  & Ax  & & \\
Set 1, Way 2 &  & Ax & Bx &  & Bv &  &  &  &  &  & Bv &  &  & Bv &  &  & Bv & & & \\
\hline
Set 2, Way 1 &  &  &  & Dx &  &  &  &  &  & Dv &  & Fx &  &  &  &  &  &  & Fv & Dx \\
Set 2, Way 2 &  &  &  &  &  & Fx &  & Ex &  &  &  &  &  &  &  & Ev & &  &  &\\
\hline
\hline
LRU & C & A & B & D & B & F & C & E & A & D & B & F & A & B & C & E & B & A & F & D \\
\hline
Set 1, Way 1 & Cx &  & Bx &  & Bv &  &  &  & Ax &  &  &  & Av &  & Cx &  &  & Ax& & \\
Set 1, Way 2 &  & Ax &  &  &  &  & Cx &  &  &  & Bx &  &  & Bv &  &  & Bv & & & \\
\hline
Set 2, Way 1 &  &  &  & Dx &  &  &  & Ex &  &  &  & Fx &  &  &  &  &  &  & Fv &\\
Set 2, Way 2 &  &  &  &  &  & Fx &  &  &  & Dx &  &  &  &  &  & Ex &  &  &  & Dx \\
\hline
\hline
MRU & C & A & B & D & B & F & C & E & A & D & B & F & A & B & C & E & B & A & F & D \\
\hline
Set 1, Way 1 & Cx &  &  &  &  &  & Cv &  & Ax &  &  &  & Av &  &  &  &  & Av & &  \\
Set 1, Way 2 &  & Ax & Bx &  & Bv &  &  &  &  &  & Bv &  &  & Bv & Cx &  & Bx & & &  \\
\hline
Set 2, Way 1 &  &  &  & Dx &  &  &  &  &  & Dv &  & Fx &  &  &  &  &  &  & Fv & Dx \\
Set 2, Way 2 &  &  &  &  &  & Fx &  & Ex &  &  &  &  &  &  &  & Ev  & & & & \\
\hline
\end{tabular}
\end{center}
\label{default}
\end{table}%
The miss rates for program 1 are:

Ideal: $\frac{11}{20}$, LRU: $\frac{15}{20}$, MRU: $\frac{11}{20}$

\begin{table}[H]
\caption{Program 2}
\begin{center}
\tiny
\begin{tabular}{|c|c|c|c|c|c|c|c|c|c|c|c|c|c|c|c|c|c|c|c|c|}
\hline
Ideal & D & F & C & B & A & A & F & C & D & D & A & B & A & B & C & E & B & A & B & D \\
\hline
Set 1, Way 1 &  &  &  & Bx & Ax & Av &  &  &  &  & Av &  & Av &  & Cx &  &  & Ax & &\\
Set 1, Way 2 &  &  & Cx &  &  &  &  & Cv &  &  &  & Bx &  & Bv &  &  & Bv &  & Bv  &\\
\hline
Set 2, Way 1 & Dx &  &  &  &  &  &  &  & Dv & Dv &  &  &  &  &  &  &  &  &  & Dv \\
Set 2, Way 2 &  & Fx &  &  &  &  & Fv &  &  &  &  &  &  &  &  & Ex & & & &\\
\hline
\hline
LRU & D & F & C & B & A & A & F & C & D & D & A & B & A & B & C & E & B & A & B & D \\
\hline
Set 1, Way 1 &  &  & Cx &  & Ax & Av &  &  &  &  & Av &  & Av &  & Cx &  &  & Ax  &  &\\
Set 1, Way 2 &  &  &  & Bx &  &  &  & Cx &  &  &  & Bx &  & Bv &  &  & Bv &  & Bv&\\
\hline
Set 2, Way 1 & Dx &  &  &  &  &  &  &  & Dv & Dv &  &  &  &  &  &  &  &  &  & Dv \\
Set 2, Way 2 &  & Fx &  &  &  &  & Fv &&&&&&&&&Ex&&&&\\
\hline
\hline
MRU & D & F & C & B & A & A & F & C & D & D & A & B & A & B & C & E & B & A & B & D \\
\hline
Set 1, Way 1 &  &  &  & Bx & Ax & Av &  &  &  &  & Av & Bx & Ax & Bx &  &  & Bv & Ax & Bx& \\
Set 1, Way 2 &  &  & Cx &  &  &  &  & Cv &  &  &  &  &  &  & Cv &&&&& \\
\hline
Set 2, Way 1 & Dx &  &  &  &  &  &  &  & Dv & Dv &  &  &  &  &  & Ex &  &  &  & Dx \\
Set 2, Way 2 &  & Fx &  &  &  &  & Fv &&&&&&&&&&&&&\\
\hline
\end{tabular}
\end{center}
\label{default}
\end{table}%
The miss rates for program 2 are:

Ideal: $\frac{9}{20}$, LRU: $\frac{9}{20}$, MRU: $\frac{12}{20}$


\end{enumerate}
\end{document}
