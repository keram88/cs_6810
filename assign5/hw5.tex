\documentclass[a4paper, 11pt]{exam}
\usepackage{titling}
\newcommand{\subtitle}[1]{%
  \posttitle{%
    \par\end{center}
    \begin{center}\large#1\end{center}
    }%
}

\usepackage{url}
\usepackage{amsmath,amsthm,enumitem,amssymb}
\usepackage{graphicx}
\usepackage{hyperref}
\usepackage{float}
\renewcommand{\labelenumii}{\roman{enumii}}

\title{Homework Assignment 2}
\subtitle{CS/ECE 6810: Computer Architecture \\
Apr 10,2018\\
Marek Baranowski}

\begin{document}
\maketitle
\begin{enumerate}
\item Virtually Indexed Caches

\item Cache and Memory Model using CACTI

\begin{enumerate}
\item For the L1 cache we have the following properties from CACTI:
\begin{verbatim}
Cache Parameters:
    Total cache size (bytes): 32768
    Number of banks: 1
    Associativity: direct mapped
    Block size (bytes): 16
    Read/write Ports: 1
    Read ports: 0
    Write ports: 0
    Technology size (nm): 32

    Access time (ns): 0.345662
    Cycle time (ns):  0.446902
    Total dynamic read energy per access (nJ): 0.0136127
    Total leakage power of a bank (mW): 12.4047
    Cache height x width (mm): 0.164353 x 0.469618 = .0772 mm^2
    Data array space efficiency: 59.6743%
    Tag array space efficiency: 67.1195%
\end{verbatim}
And for L2 we have these properties:
\begin{verbatim}
Cache Parameters:
    Total cache size (bytes): 1048576
    Number of banks: 1
    Associativity: 4
    Block size (bytes): 64
    Read/write Ports: 1
    Read ports: 0
    Write ports: 0
    Technology size (nm): 32

    Access time (ns): 1.07383
    Cycle time (ns):  1.34504
    Total dynamic read energy per access (nJ): 0.34607
    Total leakage power of a bank (mW): 335.423
    Cache height x width (mm): 1.18466 x 1.82847 = 2.167 mm^2
    Data array space efficiency: 67.3033%
    Tag array space efficiency: 76.39%
\end{verbatim}
There are a few things to note here: L2 access times are about 3 times slower
than L1 access times. But this comes at a somewhat significant cost where the
L2 cache is using about 28 times more energy than the L1 cache. However, this is
to be expected as the L2 needs at a minimum 32 times the number of
transistors that L1 requires due to their differing sizes. This result actually
implies the L2 cache is slightly more efficient with leakage power per stored
bit than the L1 cache.

The dimensions of both caches are notable: L1 is more rectangular,
whereas L2 is more square.  Playing with the parameters of each cache,
this disparity is due to the L1 cache's small size along with its
being direct mapped and size of its cache line. Interestingly only
changing the L1 cache size and associativity leads to a configuration
with the same space efficiencies. As reported by CACTI, the L1 cache
is about 10\% less space efficent than L2. We can see this in the ratio of the
areas of the cache sizes with L2 being about 28 times larger than L1, but L2 is
storing 32 times as much information. From $\frac{28}{32}\approx87.5\%$ we can 
see directly the inefficiency of the L1 cache in terms of its area; the L2 cache
is using .875 times less area per bit.
\item 
\end{enumerate}
\end{enumerate}


\end{document}